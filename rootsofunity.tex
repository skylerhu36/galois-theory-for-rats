\documentclass[12pt]{article}
\usepackage{asymptote}
\usepackage[normalem]{ulem}
\usepackage[margin=1in]{geometry} 
\usepackage{amsmath,amsthm,amssymb,amsfonts, enumitem, float, fancyhdr, color, comment, graphicx, environ,setspace, geometry}


% Define the theorem, lemma, proposition and corollary environments


\newtheorem{example}{Example}[section]
\theoremstyle{remark}
\newtheorem{theorem}{Theorem}[section]
\theoremstyle{remark}
\newtheorem{lemma}{Lemma}[section]
\theoremstyle{remark}
\newtheorem{proposition}{Proposition}[section]
\theoremstyle{remark}
\newtheorem{corollary}{Corollary}[section]
\theoremstyle{remark}
\newtheorem{definition}{Definition}[section]
\theoremstyle{remark}

\title{Roots of Unity \& Field Theory}
\author{Skyler Hu}

\begin{document}
\maketitle
\newpage
\tableofcontents
\newpage

\begin{section}{Introduction}
    In this handout we will explore the significance of roots of unity in Galois theory, and in abstract algebra in general. Roots of unity
    have always been an intriguing object of study over the evolution of mathematics: not only are they the solution to the simple and beautiful
    polynomials \(x^n-1=0\), they also have interesting algebraic structures with various applications. In addition, they can be considered to 
    pave the way for the discovery of complex numbers, since they are the simplest complex numbers to attain in an algebraic sense.\\

    \noindent We will briefly cover properties of the roots of unity as a group, then move on to roots of unity and field theory with cyclotomic extensions
    (field extensions obtained by adjoining certain roots of unity), Galois groups and the ultimate question of solvability by radicals. We will 
    see the crucial structure the roots of unity brings to a field and the implications they have on solving a polynomial.
\end{section}

\newpage

\begin{section}{Roots of Unity: Overview}
    \begin{definition}{(Roots of Unity.)}
        The \(n\)th roots of unity are the \(n\) solutions to the polynomial \(x^n-1\) over \(\mathbb{C}\). 
    \end{definition}

    \begin{corollary}{(\(n\)th roots of unity form a group.)}
        The \(n\)th roots of unity form a cyclic group of order \(n\) under multiplication.
        \begin{proof}
            We only need to verify the group axioms for roots of unity:
            \begin{enumerate}
                \item \textbf{Closure.} Let \(\zeta_1, \zeta_2\) be two \(n\)th roots of unity. Then \((\zeta_1 \zeta_2)^n = (\zeta_1)^n (\zeta_2)^n =1\), so
                the \(n\)th roots of unity is closed under multiplication.
                \item \textbf{Associativity.} Follows from the associativity of multiplication.
                \item \textbf{Identity.} The identity is 1, which is an \(n\)th root of unity for any \(n\).
                \item \textbf{Inverse.} For \(n\)th root of unity \(\zeta\), \({\frac{1}{\zeta}}^n = \frac{1}{\zeta^n} = 1\), so the \(n\)th roots of unity are 
                closed under inverses.
                Therefore the \(n\)th roots of unity form a group. Furthermore, this group is cyclic because we know that the \(n\)th roots of unity are precisely
                \[ e^{\frac{2\pi ik}{n}} \text{ for integer }k<n,\]
                and \(e^{\frac{2\pi i}{n}}\) is a generator. Similarly, for all \(\gcd(k,n)=1\), \(e^{\frac{2\pi ik}{n}}\) is a generator for the group of 
                the \(n\)th roots of unity.\\
                \noindent The group of \(n\)th roots of unity is denoted \(\mu_n\).
            \end{enumerate}
        \end{proof}
    \end{corollary}

    \noindent Since we have shown that the roots of unity form a cyclic group, we can define the \textit{primitive roots of unity} as the set of its generators:
    \begin{definition}{(Primitive Roots of Unity.)}
        The \(n\)th primitive roots of unity are the generators of \(\mu_n\).
    \end{definition}

    \noindent Some important results concerning \(\mu_n\):
    \begin{corollary}{(Properties.)}
        \begin{enumerate}
            \item If \(\zeta\) is a \(d\)th root of unity and \(d\mid n\), \(\zeta\) is also an \(n\)th root of unity. Hence \(\mu_d \leq \mu_n\) for all \(d\mid n\).
            \item Conversely, if \(\zeta\) is an \(n\)th root of unity which is also a \(d\)th root of unity (\(d\leq n\)), then \(d \mid n\). The order of \(\zeta\)
            in \(\mu_n\) is \(d\).
        \end{enumerate}
    \end{corollary}
\end{section}

\newpage
\begin{section}{Cyclotomic Polynomials}
    \begin{definition}{(Cyclotomic Polynomials.)}
        The \(n\)th cyclotomic polynomial \(\phi_n(x)\) is the polynomial whose roots are the primitive \(n\)th roots of unity:
        \[\phi_n(x) = \prod_{\zeta \text{ is a primitive nth root of unity}}^{}(x-\zeta).\]
    \end{definition}

    \noindent Some properties of the cyclotomic polynomial:
    \begin{theorem}
        The \(n\)th cyclotomic polynomial \(\phi_n(x)\) is a monic polynomial in \(\mathbb{Z}[x]\) of degree \(\varphi(n)\), where \(\varphi\) denotes Euler's
        Totient Function, the number of naturals less than or equal to \(n\) that are coprime with \(n\).
    \end{theorem}
    \begin{proof}
        It is clear that \(\phi_n(x)\) is a monic polynomial. Since there are \(\varphi(n)\) primitive \(n\)th roots of unity, the degree of \(\phi_n(x)\) is 
        \(\varphi(n)\). We only need to prove that the coefficients of \(\phi_n(x)\) lie in \(\mathbb{Z}\).\\
        \noindent\textit{Proof by Induction.}\\
        \noindent\textbf{Base Case.} The 1st cyclotomic polynomial is \(x-1\), which clearly has all coefficients in \(\mathbb{Z}\).\\
        \noindent\textbf{Inductive Hypothesis.} Suppose that the statement is true for \(1\leq d \leq n\). We have 
        \[x^n-1 = \phi_n(x)(\prod_{d \mid n, d<n}^{}\phi_d(x)).\]
        Now we know that \(x^n-1\) and \(f(x) = \prod_{d \mid n, d<n}^{}\phi_d(x)\) all have integer coefficients, and \(f(x)\) divides \(x^n-1\) in \(\mathbb{Q}(\zeta)\),
        the field extension of the \(n\)th roots of unity over the rationals, so \(f(x)\) divides \(x^n-1\) in \(\mathbb{Q}\) and therefore in \(\mathbb{Z}\) by Gauss's Lemma.
    \end{proof}

    \begin{theorem}{(Cyclotomic Polynomial is irreducible)}
        The \(n\)th cyclotomic polynomial \(\phi_n(x)\) is a monic \textit{irreducible} polynomial in \(\mathbb{Z}[x]\) of degree \(\varphi(n)\).
    \end{theorem}
    \begin{proof}
        We only need to show that \(\phi_n(x)\) is irreducible over \(\mathbb{Z}\).\\
        \noindent Suppose, for a contradiction, that \(\phi_n(x) = f(x)g(x)\) for monic \(f(x), g(x) \in \mathbb{Z}[x]\). Let \(\zeta\) be a root of \(\phi_n(x)\) and \(f(x)\) be 
        the minimal polynomial of \(\zeta\) over \(\mathbb{Z}\). Let \(p\) be a prime not dividing \(n\). Then \(\zeta^p\) is a primitive \(n\)th root of unity, 
        and is therefore a root of either \(f(x)\) or \(g(x)\).\\
        \noindent\textbf{Case 1.} \(\zeta^p\) is a root of \(g(x)\).\\
        \noindent Then \(\zeta\) is a root of \(g(x^p)\). Since \(f(x)\) is the minimal polynomial of \(\zeta\), \(f(x)\) divides \(g(x^p)\) in \(\mathbb{Z}[x]\):
        \[g(x^p) = f(x) h(x) text{ for some } h(x) \in \mathbb{Z}[x].\]
        Reducing mod \(p\) gives us \(\bar{g(x^p)} = \bar{f(x) \bar{h(x)}}\) in \(F_p\). Using the fact that \(\bar{g(x^p)} = (g(x))^p\):
        \[(g(x))^p = \bar{f(x)} \bar{h(x)} \text{ in } F_p.\]
        Therefore \(g(x)\) and \(f(x)\) have a factor in common in \(F_p[x]\). \\
        \noindent Similarly, we can reduce the cyclotomic polynomial mod \(p\):
        \[\bar{\phi_n(x)} = \bar{f(x)} \bar(g(x)) \text{ in } F_p[x].\]
        Since \(f(x)\) and \(g(x)\) share a root in \(F_p[x]\), \(\phi_n(x)\) has a multiple root, contradicting the separability of \(x^n-1\). This case gives a contradiction and is 
        therefore invalid.\\

        \noindent\textbf{Case 2.} \(\zeta^p\) is a root of \(f(x)\).\\
        \noindent This applies to every \(\zeta\) a root of \(\phi_n(x)\), so \(\phi_n(x) = f(x)\), and \(\phi_n(x)\) is irreducible.
    \end{proof}

    We can conclude from the above theorem:
    \begin{corollary}
        The degree over \(\mathbb{Q}\) of the cyclotomic field (field containing all the \(n\)th roots of unity) \(\mathbb{Q}(\zeta_n)\) is \(\varphi(n)\):
        \[[\mathbb{Q}(\zeta_n) : \mathbb{Q}] = \varphi(n)\].
    \end{corollary}
    \begin{proof}
        The minimal polynomial of any primitive \(n\)th root of unity is \(\phi_n(x)\) with degree \(\varphi(n)\). Note that the primitive \(n\)th roots of unity generate the \(n\)th 
        roots of unity, so \(\mathbb{Q}\) adjoined the primitive \(n\)th roots of unity is exactly \(\mathbb{Q}\) adjoined the \(n\)th roots of unity. Since \(phi_n(x)\) is 
        separable, the degree of \(\mathbb{Q}(\zeta_n)\) is exactly the number of primitive \(n\)th roots of unity, which is \(\varphi(n)\).
    \end{proof}
\end{section}

\end{document}