\documentclass{article}

\usepackage{amsmath}
\usepackage{amsfonts}
\usepackage{asymptote}
\usepackage[normalem]{ulem}

\setcounter{section}{-1}

\title{Galois Theory for Rats}
\author{Skyler Hu}

\begin{document}
\maketitle
\newpage
\tableofcontents
\newpage

\begin{section}{Before we Start}
    \begin{subsection}{Galois Theory for Rats: Definition}
        This is intended to be a handout for people interested in Galois Theory. It is not, in any way, meant
        to be a full, concise course on Galois Theory. Rather, I have tried to introduce topics and ideas in a 
        way understandable to people new to abstract algebra.\\
        \\
        The handout assumes basic knowledge on groups, rings and fields: readers should at least know the definition
        of groups, rings and fields, know some common small groups, and have some intuition on how they operate. To give
        an idea of this level of proficiency, a dedicated math nerd should be able to achieve the level within 1-2 months 
        of studying algebra. 
    \end{subsection}\\
    \\

    \begin{subsection}{Motivation}
        As you may know, Galois theory is considered one of the most beautiful branches of mathematics (or at least of 
        algebra). It addresses the longstanding, fundamental problem of solvability by radicals -- why the general equation of 
        fifth degree is not solvable by radicals, and sheds light on many hidden aspects of polynomials and roots. In addition,
        Galois theory is applicable to various subjects of algebra, including group theory, topology and algebraic number theory.\\
        \\
        The story behind Galois theory is also one of legend and tragedy. It's a well known anecdote among math people that Evariste
        Galois, the young genius behind Galois theory, died under mysterious circumstances in a duel at the age of 20. During his lifetime,
        he was a revolutionary mathematician, a literal revolutionary (he was an avid Republican during the French Revolution), and an ardent,
        impassioned soul. Studying Galois theory might give us some insight into his short, brilliant life and mathematics.\\
        \\
        I first encountered Galois theory in SUMaC 2025. I had been reading on mathematical history prior to the camp, and the idea of solvability
        through algebraic methods, as well as Galois' life, was fascinating. During this fantastic month I gained a basic understanding
        of Galois theory, as well as an overwhelming number of inside math jokes concerning it, and this has motivated me to share this 
        theory, as well as the experience of learning something new, with the math community.
    \end{subsection}
    \\

    \begin{subsection}{Acknowledgements}
        I would like to give thanks to all the brilliant rats I met. Also bro, gang and 
        chief, Wilbur watermelon, and whoever supplies SUMaC with all the snacks every day.
    \end{subsection}


\end{section}
\newpage



\end{document}