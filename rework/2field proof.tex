\documentclass[12pt]{article}
\usepackage[margin=1in, bottom = 1.5in]{geometry} 
\usepackage{amsmath,amsthm,amssymb,amsfonts, enumitem, float, fancyhdr, color, comment, graphicx, environ,setspace, subcaption, lastpage, tabularx, booktabs, caption}
\captionsetup[subfigure]{width=0.85\linewidth}
\pagestyle{fancy}
\fancyhf{}
\lhead{Galois Theory for Rats}  
\rhead{Field Extensions}
\fancyfoot[R]{Page \thepage\ of \pageref{LastPage}}
\newlength{\myfootheight}
\setlength{\myfootheight}{35pt} 
\renewcommand{\footrulewidth}{0.4pt}
\setlength{\headheight}{35pt}
\pagenumbering{arabic}


% Define the theorem, lemma, proposition and corollary environmentsc
\newtheorem{proposition}{Proposition}[section]
\theoremstyle{definition}
\newtheorem{corollary}{Corollary}[section]
\theoremstyle{definition}
\newtheorem{definition}{Definition}[section]
\theoremstyle{definition}
\newtheorem{example}{Example}[section]
\theoremstyle{definition}
\newtheorem{theorem}{Theorem}[section]
\theoremstyle{definition}
\newtheorem{lemma}{Lemma}[section]
\theoremstyle{definition}

\AtBeginDocument{%
  \mathchardef\stdcomma=\mathcode`,
  \mathcode`,="8000
}
\begingroup\lccode`~=`, \lowercase{\endgroup\def~}{\stdcomma\,}

\title{Galois Theory for Rats - Field Representation}
\author{Skyler Hu}

\begin{document}
\linespread{1.05}
\maketitle
\newpage
\tableofcontents
\newpage

\begin{section}{Introduction \& Motivation}
    In this handout on \textit{Field Representation}, we will introduce and prove a single result: there exists
    exactly 1 field of order \(p^n\) for any prime \(p\) and any positive integer \(n\).\\
    \break
    There are some reasons to why this result is important. First, it's simplicity: readers familiar with group 
    theory will know that one of the most (and arguably \textit{the} most) important topics when working with 
    groups is \textit{Representation Theory}, describing the types of groups of given order. Representation Theory
    is still an active area of research today, and its core project -- the classification of all finite simple groups
    -- is recognized as one of the most prodigious and monumental efforts in modern mathematics. However, the mirror
    idea when it comes to fields -- classifying all the fields -- takes the length of this handout and a lunch break.\\
    \break
    The insight this comparison provides is profound. In comparing different algebraic structures, fields has only 5 (6?)
    more structural constraints than groups, but these couple additional axioms scale down the difficulty of the same
    question by an unimaginable extent. In general, it's safe to consider that the simplicity of a task increases with
    more constraints (contrary to intuition): a linear equation with two variables has infinite solutions, but two linear
    equations with two variables only have one (or occasionally zero); there are hundreds of thousands of people named Skyler,
    but there is only one Skyler who wrote this handout.
\end{section}

\begin{section}{Proof}
    \begin{theorem}
      There exists a field of order \(p^n\) for all prime \(p\) and positive integer \(n\). This field is unique up to 
      isomorphism.
      \begin{proof}
        \noindent\textbf{1. Existence:} Let \(n>0\) be any integer and consider \(x^{p^n}-x \in F_p[x]\). This polynomial
        is separable (it is equal to \(x(x^{p^n-1})-1\)) and therefore has \(p^n\) distinct roots.\\
        Consider the splitting field of \(x^{p^n}-x\) over \(F_p[x]\). Let \(\alpha, \beta\) be roots of \(x^{p^n}-x\). Since \(\alpha^{p^n} = \alpha\),
        \(\beta_{p^n} = \beta\),
        \[(\alpha\beta)^{p^n} = \alpha^{p^n} \beta^{p^n} = \alpha\beta, \alpha^{p^n} = \alpha^{-1} \text{ and } (\alpha+\beta)^{p^n} = \alpha^{p^n} + \beta^{p^n} = \alpha+\beta,\]
        so the set \(F\) containing all the roots of \(x^{p^n} - x\) is closed under addition, multiplication and inverse, and \(F\) is a field. Since \(F\) is a subfield of 
        the splitting field of \(x^{p^n}-x\), \(F\) \textit{is} the splitting field.\\
        Since \(\mid F\mid = p^n\), there exists finite fields of order \(p^n\) for any prime \(p\) and integer \(n\). We can obtain a field of order \(p^n\) by finding the 
        splitting field of the polynomial \(x^{p^n}-x\) over \(F_p\).\\
        In addition, since \([F:F_p] = n\), there exists finite fields of degree \(n\) over \(F_p\) for any \(n>0\).\\
        \break
        \noindent\textbf{2. Uniqueness:} Let \(F\) be a field of characteristic \(p\). If \([F:F_p] = n\) where \(F_p\) is the prime
        subfield of \(F\), then \(\mid F \mid = p^n\). Since the multiplicative group \(F^{\times}\) has order \(p^n-1\), \(\alpha^{p^n-1} = 1\) for any \(\alpha \in F^{\times}\),
        giving \(\alpha^{p_n} = \alpha\). Then \(\alpha\) is an element in the splitting field of \(x^{p^n}-x\) over \(F_p\), which has precisely order \(p^n\).
        Then \(F\) is a splitting field of \(x^{p^n}-x\), and is unique up to isomorphism by the properties of splitting fields.\\
        \break
        \noindent\textit{Conclusion.} There exists a unique field of order \(p^n\) for prime \(p\) and positive integer \(n\).
      \end{proof}
    \end{theorem}
\end{section}

\end{document}