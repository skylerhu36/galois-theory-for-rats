\documentclass[12pt]{article}
\usepackage[inline]{asymptote}
\usepackage[normalem]{ulem}
\usepackage[margin=1in]{geometry} 
\usepackage{amsmath,amsthm,amssymb,amsfonts, asymptote, enumitem, float, fancyhdr, color, comment, graphicx, environ,setspace, geometry}


% Define the theorem, lemma, proposition and corollary environments
\newtheorem{corollary}{Corollary}[section]
\theoremstyle{remark}
\newtheorem{definition}{Definition}[section]
\theoremstyle{remark}
\newtheorem{example}{Example}[section]
\theoremstyle{remark}
\newtheorem{theorem}{Theorem}[section]
\theoremstyle{remark}
\newtheorem{lemma}{Lemma}[section]
\theoremstyle{remark}
\newtheorem{proposition}{Proposition}[section]
\theoremstyle{remark}

\title{Straightedge and Compass Constructions}
\author{Skyler Hu}

\begin{document}
\maketitle
\newpage
\tableofcontents
\newpage

\begin{section}{Introduction to Straightedge \& Compass Construction}
    Straightedge and compass construction is defined as graphing with only a straightedge (item with no length label and a straight edge) and 
    a compass. This allows only 3 operations given two points on a plane, \(A\) and \(B\):
    \begin{enumerate}
        \item Connecting the line segment \(AB\);
        \item Drawing a circle with center \(A\) through point \(B\) (i.e. the radius is \(AB\));
        \item Marking any point of intersection of known lines and circles.
    \end{enumerate}

    It is known from elementary geometry that the following can be achieved with straightedge and compass construction:
    \begin{enumerate}
        \item Constructing the perpendicular bisector of a given line segment (thus bisecting any line segment);
        \item Constructing the angle bisector of any given angle; (this allows us to construct a right angle)
        \item Constructing the sum or difference of two given line segments (achieved by drawing 2 circles);
        \item Constructing a line through a given point that's parallel to a given line.
    \end{enumerate}

    Though appearing to be elementary geometry, straightedge and compass constructions have important implications for field theory and 
    abstract algebra, specifically quadratic field extensions and cyclotomic extensions (see \textit{Roots of Unity}). In such construction problems,
    we are interested in what lengths are constructible given a line segment of unit length 1. We can define the following from this idea:

    \begin{definition}{(Constructible Numbers.)}
        A real number \(x\) is \textit{constructible} if a line segment of length \(x\) can be constructed via straightedge and compass only from a 
        line segment of length 1.
    \end{definition}

    Straightedge and compass construction problems are concerned with characterizing the constructible numbers as a subset of \(\mathbb{R}\). As we will 
    see in the following sections, this characterization is achieved through field-theoretic proofs and helps us solve millennia-old problems on geometric 
    constrcution.
\end{section}

\begin{section}{Three Classic Construction Problems}
    These problems have existed for as long as people have studied mathematics, and are recently solved with straightedge and compass construction theory:
    \begin{enumerate}
        \item \textit{Duplicating the Cube} Is it possible to construct a cube with twice the volume of a given cube?
        \item \textit{Trisecting the Angle} Is it possible to trisect an angle using straightedge and compass only?
        \item \textit{Squaring the Circle} Is it possible to construct a square with the area of a given circle?
    \end{enumerate}
\end{section}

\begin{section}{Basic Construction Rules}
    \begin{proposition}{(The Four Operations via Construction.)}
        Given line segments of length \(a,b\), we can construct the following:
        \[\{a+b, a-b, ab, a/b, \sqrt{a}\}.\]
        
        \begin{proof}
            \textbf{Addition and Subtraction.} This is known from elementary geometry.\\
            \noindent\textbf{Multiplication and Division.} 
            \begin{center}
                \begin{asy}
                    settings.outformat = "pdf";
                    size(8cm);

                    // Define vertices for outer triangle ADE
                    pair D = (0,0);
                    pair E = (6,0);
                    pair A = (1.5,4);

                    // Choose B on AD and C on AE such that ABC is similar to ADE
                    real scale = 0.6; // scaling factor for inner triangle
                    pair B = A + scale*(D - A);
                    pair C = A + scale*(E - A);

                    // Draw outer triangle ADE
                    draw(D--A--E--cycle, black+1pt);

                    // Draw inner triangle ABC
                    draw(A--B--C--cycle, black+1pt);

                    // Label segments
                    label("$1$", (A+B)/2, S);  // BC = 1
                    label("$a$", (A+C)/2, NE); // AC = a
                    label("$b$", (A+D)/2, NW); // AD = b
                    label("$ab$", (A+E)/2, NE); // AE = ab

 
                \end{asy}
            \end{center}

            % I hate fucking asymptote and maybe I'll do this another time

            \noindent \textbf{Square Roots.}
            \begin{lemma}
                In a right triangle, the square of the height on the hypotenuse is equal to the product of the two line segments to the left and right
                of the height.
                % insert image here
            \end{lemma}

            % insert image here
        \end{proof}
    \end{proposition}

    We can characterize with Proposition 3.1 the real numbers that are constructible:\\
    \noindent \textbf{Intersection of 2 straight lines.} Consider the general form of two straight lines:
    \begin{align*}
        a_1x+b_1y+c_1 &= 0\\
        a_2x+b_2y+c_2 &= 0 
    \end{align*}
    By substituting \(y = -\frac{c_2+a_2x}{b}\), this amounts to solving a linear equation in terms of \(x\), which gives a linear combination of the coefficients. 
    Since \(a_1, a_2, b_1, b_2, c_1, c_2\) are known from two points on the line within the field of constructible numbers \(F\), it could be said that solving for 
    \((x,y)\) gives solutions that are also in \(F\).\\

    \noindent\textbf{Intersection of a straight line and a circle.} The general form is 
    \begin{align*}
        ax + by + c &= 0\\
        (x-h)^2 + (y-k)^2 &= r^2\\
    \end{align*}
    Again by taking the linear substitution \(y = -\frac{c+ax}{b}\), this amounts to solving a quadratic in terms of \(x\). The solution is therefore in a quadratic 
    extension of \(F\).\\

    \noindent\textbf{Intersection of two circles.} The general form of two circles Is
    \begin{align*}
        (x-h_1)^2 + (y-k_1)^2 &= r_1^2\\
        (x-h_2)^2 + (y-k_2)^2 &= r_2^2
    \end{align*}
    To see that this is equivalent to the intersection of a circle and a straight line, take the difference of the two equations to get 
    \begin{align*}
        (x-h_1)^2 + (y-k_1)^2 &= r_1^2\\
        2(h_1-h_2)x + 2(k_1-k_2)y + (h_1^2-h_2^2) + (k_1^2+k_2^2) &= r_1^2 - r_2^2
    \end{align*}
    This is the same as the previous case, and solving for \(x,y\) gives us at worst a quadratic extension over \(F\).\\
    Since all constructions give either elements already in \(F\) or elements in a quadratic extension over \(F\), we can conclude that all constructible 
    numbers are in an extension of degree \(2^k\) over \(\mathbb{Q}\):

    \begin{theorem}{(What numbers are constructible?)}
        If real number \(\alpha\) is obtained from base field \(F\) through straightedge and compass constructions, then \([F(\alpha) : F] = 2^k\).
    \end{theorem}

    We can derive the following:

    \begin{theorem}{(Classic construction problems)}
        None of the three classic construction problems: duplicating the cube, squaring the circle and trisecting the angle is achievable by 
        straightedge and compass constructions.

        \begin{proof}
            \textbf{Duplicating the Cube.} This requires us to construct a cube of side length \(\sqrt[3]{2}\). Since \([\mathbb{Q}(\sqrt[3]{2}):\mathbb{Q}] = 3\),
            and 3 is not a power of 2, \(\sqrt[3]{2}\) is not constructible.\\
            \noindent\textbf{Squaring the Circle.} Given a circle of radius \(r\), we must construct a square with side length \(\sqrt{\pi}r\), which is constructible
            if and only if \(\pi\) is constructible. It is known that \(\pi\) is not constructible -- there exists a proof that \(pi\) is transcendental, and for a 
            number to be constructible it must be algebraic (see Theorem 3.1). Therefore it is impossible to construct a square with the same area of a given circle
            using straightedge and compass.\\
            \noindent\textbf{Trisecting the angle.} Given angle \(\theta\), \(\cos\theta\) is constructible since it's the horizontal distance from the point of intersection
            of angle \(\theta\) and the unit circle to the \(y\)-axis. The problem is then rewritten: given \(\cos\theta\), is \(\cos\frac{\theta}{3}\) constructible?
            We show that \(\cos\frac{\theta}{3}\) is not always constructible by considering a counterexample: \(\theta = 60^{\deg}\). Then the question is whether or not 
            \(\cos20^{\deg}\) is constructible. The triple angle formula for cosines states that 
            \[\cos60^{\deg} = 4\cos^3 20^{\deg} - 3\cos20^{\deg},\]
            so, taking \(x = \cos20^{\deg}\), we want to solve the cubic 
            \[4x^3-3x-\frac{1}{2} = 0, \text{ or } 8x^3 - 6x - 1= 0.\]
            This is equivalent to \((2x)^3 - 3(2x) -1=0\). Taking \(y=2x\), the equation becomes \(y^3-3y-1=0\). It could be easily verified that this equation is irreducible
            over \(\mathbb{Q}\) since it has no rational roots, so \([\mathbb{Q}(y):\mathbb{Q}] = 3\), which is not a power of 2, so \(\cos20^{\deg}\) is not constructible.
        \end{proof}
    \end{theorem}

\end{section}

\end{document}